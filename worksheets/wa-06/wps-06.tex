\documentclass[11pt]{article}

\usepackage{listings}
\usepackage{fancyhdr}
\usepackage[margin=.8in]{geometry}
\usepackage{amsmath}
\usepackage{enumitem}

\linespread{1.2}
\setlength{\parindent}{0pt}
\setlength{\tabcolsep}{15pt}

% ===========================================================================
% Header / Footer
% ===========================================================================
\pagestyle{fancy}
\lhead{\scriptsize  CSC 212: Data Structures and Abstractions - Spring 2018}\chead{}\rhead{\scriptsize Weekly Problem Set \#6}
\lfoot{}\cfoot{\scriptsize \thepage~of~\pageref{r:lastpage}}\rfoot{}
\renewcommand{\headrulewidth}{0.3pt}
\renewcommand{\footrulewidth}{0.3pt}

% ===========================================================================
% ===========================================================================
\begin{document}
\thispagestyle{empty}

% ===========================================================================
\begin{center}
    {\Large\bf CSC 212: Data Structures and Abstractions}\\
    \medskip
    {\Large\bf Spring 2018}\\
    \medskip
    {\Large\bf University of Rhode Island}\\
    \bigskip
    {\Large\bf Weekly Problem Set \#6}
\end{center}

Due Thursday 3/8 before class. Please turn in neat, and organized, answers hand-written on standard-sized paper \textbf{without any fringe}. At the top of each sheet you hand in, please write your name, and ID.
The only library you're allowed to use in your answers is \verb|iostream|.

\section{Recursion}
Solve the following using only recursion.
\begin{enumerate}
    \item Reverse the elements of an array in place. Matching the following function signature: 
    
        \verb|void reverse_array(int* arr, int n);|

    \item Find and return the maximum of a given array. Matching the following function signature: 
    
        \verb|int max_array(int* arr, int n);|

    \item Write a function to print triangles to \verb|std::cout| that takes three positive integers: $a$, $b$, $c$ as input. The function should print the \verb|+| character $a$ times, then $a+c$ times, then $a+c+c$ times, and so on. This pattern should repeat until the line is $b$ characters long. At that point, the pattern is repeated backwards. For example calling \verb|draw_triangle(4, 7, 1)| will output: (where the dollar symbol is the bash command prompt)
    \begin{verbatim}
        ++++
        +++++
        ++++++
        +++++++
        +++++++
        ++++++
        +++++
        ++++
    \end{verbatim} 

    \item Recursively multiply two numbers together, \emph{without using the * operator}. Matching the following function signature:

        \verb|int multiply(int a, int b);|

    \item Suffix summation is the sum from $n$ to $n-s$. Matching the following function signature:

        \verb|int suffix_sum(int n, int s);| 

    For example, the suffix sum of $n=5, s=2$ is $(5+4+3) = 12$
\end{enumerate}

\label{r:lastpage}

\end{document}
    
