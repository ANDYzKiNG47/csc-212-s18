\documentclass[11pt]{article}

\usepackage{listings}
\usepackage{fancyhdr}
\usepackage[margin=.8in]{geometry}
\usepackage{amsmath}
\usepackage{enumitem}

\linespread{1.3}
\setlength{\parindent}{0pt}

% ===========================================================================
% Header / Footer
% ===========================================================================
\pagestyle{fancy}
\lhead{\scriptsize  CSC 212: Data Structures and Abstractions - Spring 2018}\chead{}\rhead{\scriptsize Weekly Problem Set \#4}
\lfoot{}\cfoot{\scriptsize \thepage~of~\pageref{r:lastpage}}\rfoot{}
\renewcommand{\headrulewidth}{0.3pt}
\renewcommand{\footrulewidth}{0.3pt}

% ===========================================================================
% ===========================================================================
\begin{document}
\thispagestyle{empty}

% ===========================================================================
\begin{center}
    {\Large\bf CSC 212: Data Structures and Abstractions}\\
    \medskip
    {\Large\bf Spring 2018}\\
    \medskip
    {\Large\bf University of Rhode Island}\\
    \bigskip
    {\Large\bf Weekly Problem Set \#4}
\end{center}

Due Wednesday 2/21 before lab. Please turn in neat, and organized, answers hand-written on standard-sized paper \textbf{without any fringe}. At the top of each sheet you hand in, please write your name, and ID.
The only library you're allowed to use in your answers is \verb|iostream|.

\begin{enumerate}[leftmargin=*]

    % R-1.7
    \item Rank the following functions by their asymptotic growth rate in ascending order.  In your solution, group those functions that are big-Theta of one another (all $\log$ functions are base 2):
    \begin{equation*}
        \begin{array}{ccccc}
            6 \cdot n\log n & 2^{100} & \log \log n & \log^2 n & 2^{\log n} \\
            2^{2^n} & \lceil\sqrt{n}\rceil & n^{0.01} & 1/n & 4n^{3/2} \\
            4^n & n^3 & n^2\log n & 4^{\log n} & \sqrt{\log n} \\
        \end{array}
    \end{equation*}

    % R-1.3
    \item Algorithm \verb|algo1| uses $10n\log n$ operations, while algorithm \verb|algo2| uses $n^2$ operations.  What is the value of $n_0$, such that \verb|algo1| is better than \verb|algo2| for $n\ge n_0$.

    % R-1.12 R-1.14 R-1.15
    \item For each of the following, give both a big-Oh characterization in terms of $n$, and an exact characterization (count additions and multiplications):
        \begin{enumerate}
        \item
        \begin{verbatim}
        EX: For the following, the big-Oh characterization is: O(n), 
        the exact characterization is n.
        s = 1
        for i = 1 to n do
            s = s * i
        \end{verbatim}
        \item
        \begin{verbatim}
        s = 1
        for i = 1 to 4n do
            s = s * i
        \end{verbatim}
        \item
        \begin{verbatim}
        s = 1
        for i = 1 to n*n*n do
            s = s * i
        \end{verbatim}
        \item
        \begin{verbatim}
        s = 0
        for i = 1 to 4n do
            for j = 1 to i do
                s = s + i
        \end{verbatim}
        \item
        \begin{verbatim}
        s = 0
        for i = 1 to n*n do
            for j = 1 to i do
                s = s + i
        \end{verbatim}
        \item
        \begin{verbatim}
        s = 1
        for i = 1 to n do
            for j = 1 to n do
                for k = 1 to n do
                    s = s * i
        \end{verbatim}
    \end{enumerate}

    % C-1.8
    \item Suppose you run two algorithms, \verb|P| and \verb|Q|, on many randomly generated data sets.  \verb|P| is an $O(n \log n)$-time algorithm and \verb|Q| is an $O(n^2)$-time algorithm.  After your experiments you find that if $n<100$, \verb|Q| actually runs faster, and only when $n\ge 100$, \verb|P| is faster.  Explain why this scenario is possible, including numerical examples.

\end{enumerate}

The following is considered optional:
\begin{enumerate}
    
    \item Given an array \verb|A|, of $n$ integers, describe a method to find the longest subarray of \verb|A| such that all the numbers in that subarray are in sorted order.  What is the running time of your algorithm?

\end{enumerate}

\label{r:lastpage}

\end{document}
    
